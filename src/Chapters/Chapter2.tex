% Chapter Template

\chapter{Problem Statement} % Main chapter title

\label{Chapter2} % Change X to a consecutive number; for referencing this chapter elsewhere, use \ref{ChapterX}

Ad-driven models fall prey to serving the needs of advertisers and looking things through their scope. Most online content review platforms do not distribute value to their creators and users with fairness in mind. Some of the existing platforms have tried to solve these problems with crypto monetization and extra rewards to the creators and users. While this is an interesting approach, their discovery mechanisms can be easily manipulated by bots. The proposed concept of NVT Network, is trying to revolutionize the online reviewing industry by launching a solution that solves the following problems:

\begin{enumerate}
\item Unequal Distribution and Platform Value.

Users are contributing content and data to online platforms, but these platforms are capturing an unequal amount of the value created by users' content. Creators and users deserve to be rewarded more efficiently for the services they provide or data they contribute to an online network. Blockchain networks like NVT, provide a new way to share network value more fairly to everyone.

\item Ineffective Product Discovery.

People need a reliable way to find quality services and products. Old and out of date platforms, which are promoting creators and their products, have issues with manipulation and paid upvoting bots that harm content ranking quality, review and board spamming. This makes it hard for users to quickly find good quality services.

\item Centralized Product Review Moderation.

Moderators are an important part of online communities as they decide which listings should get listed on the platform, monitor the actions of the users and apply penalties if needed, influence and moderate content promotion, etc. This centralized control means that other users do not have a formal way to dispute decisions made by moderators. NVT Network is exploring ways to change this.

\end{enumerate}

%----------------------------------------------------------------------------------------
%	SECTION 1
%----------------------------------------------------------------------------------------

\section{Type of Value}
\subsection{Tangible Value}

All exchanges including goods, services or revenues and transactions are considered tangible values. The transactions might involve contracts, invoices, request for proposals or confirmations, return receipts of orders and payments. Products or services that are expected to generate revenue are also included in the tangible value flow. In government agencies these would be mandated activities. In civil society organizations these would be formal commitments to provide resources or services.

%-----------------------------------
%	SUBSECTION 1
%-----------------------------------
\subsection{Intangible Value}

Two primary subcategories are included in intangible value: knowledge and benefits. Intangible knowledge exchanges include strategic information, planning knowledge, process knowledge, technical know-how, collaborative design and policy development; which support the product and service tangible value network. Intangible benefits are also considered favors that can be offered from one person to another. Examples include offering political or emotional support to someone. Another example of intangible value is when a research organization asks someone to volunteer their time and expertise to a project in exchange for the intangible benefit of prestige by affiliation.

All biological organisms, including humans, function in a self-organizing mode internally and externally. That is, the elements in our bodies down to individual cells and DNA molecules work together in order to sustain us. However, there is no central ”boss” to control this dynamic activity. Our relationships with other individuals also progress through the same circular free flowing process as we search for outcomes that are best for our well-being. Under the right conditions these social exchanges can be extraordinarily altruistic. Conversely, they can also be quite self-centered and even violent. It all depends on the context of the immediate environment and the people involved.


%-----------------------------------
%	SUBSECTION 2
%-----------------------------------

\subsection{a Non-Linear Approach}
Often value networks are considered to consist of groups of companies working together to produce and transport a product to the customer. Relationships among customers of a single company are examples of how value networks can be found in any organization. Companies can link their customers together by direct methods like the telephone or indirect methods like combining customer’s resources together. The purpose of value networks is to create the most benefit for the people involved in the network. The intangible value of knowledge within these networks is just as important as a monetary value. In order to succeed, knowledge must be shared to create the best situations or opportunities. Value networks are how ideas flow into the market and to the people that need to hear them.

%----------------------------------------------------------------------------------------
%	SECTION 2
%----------------------------------------------------------------------------------------

\section{An Alternative to Mining}

Referring a friend or distributing knowledge and informational reviews about products, networks or groups to other social media channels, is essential to the community. These points can be used in promotion of a listing. Product-discovery can be compared to proof of work. Mining requires electricity, and the electricity bills in this case can be paid either by NVT tokens or labor for the growth of the community. The process of product-evaluation and the prediction model is described in Section 3.1.

The basic principle behind this idea is that any user can assign NVT tokens to creators, networks or listings according to the judgment and community review of the product. Additionally, any user can assign NVT tokens to the community and user or explorer base according to the foresight of each voter.

%----------------------------------------------------------------------------------------

\section{Core Transparency}

Because the number of fraudelent projects is too high, it can be hard for investors to decide which one is worthy and has legitimacy. The NVT Team believes that investors and token sale organizers alike, will benefit from some form of increased financial transparency. It will give the investors the confidence needed that a startup is directing the funds given toward the development of the project and its vision. Quality startups likely, attract more funds when properly demonstrate a wisely constructed financial plan to the investors.

This has led the NVT Team to decide that it is imperative to lock its tokens for a vesting period of 12 months. Furthermore, a special report page will be made at the \href{https://nvtnetwork.com/}{NVT Foundation Website}, including the addresses of everyone involved with the team and their wage transactions. Lastly, the NVT Team will always be available to community review and it is going to be transparent with its expenses and all other financial data.

%----------------------------------------------------------------------------------------

\section{Network Value Transactions}

In traditional finance, ratio analysis is one of the most widely used valuation methods. Lacking the detail of other valuation approaches, such as DCF analysis, ratio-based valuation is much faster and is still a good proxy of fair value. It also allows one to easily track asset price dynamic over long periods of time as well as compare different assets to each other. Over the course of the last year, a new study of cryptoeconomic ratio analysis emerged. One of the most widely known ratios is Network Value to Transactions, or NVT.

In a traditional PE ratio, the earnings metric in the denominator is used as a proxy for the underlying utility of the company created for the shareholders. While cryptoassets do not have earnings, one can argue that the total value of transactions flowing through the network is a proxy for how much utility users derive from the chain.

It is worth highlighting that Daily Transaction Volume in NVT takes into account only on-chain transactions. All the trading activity that happens on exchanges and is, for the most part, speculative is not included in this volume. Some researchers argue that NVT can be successfully used to detect bitcoin price bubbles when valuation is not supported by fundamentals and differentiate them from consolidations. If we analyze historical data, the spike in NVT follows the bubble with a considerable lag of a few months. Peak NVT coincides with the middle of a correction period. NVT is neither predictive (does not precede the overvaluation), nor descriptive (does not coincide with it). You can only detect the bubble a few months after it bursts.

We claim that this refined NVT ratio is a better descriptive metric of generic bubbles. Conceptually, this makes sense. Given that Transaction Volume in NVT is a proxy for fundamental utility value of the network, a 90-day Moving Average is a better proxy for long-term fundamental value than a 28-day Moving Average.

Let’s now look at the recent bitcoin price performance using the refined NVT ratio in more detail. From January until mid-December 2017, bitcoin has appreciated almost 20x. For the most part of this rally, though, NVT ratio has stayed in the Green Zone. However, in December when price reached almost \$20k, NVT went into the Yellow for a few days. This rapid appreciation was shortly followed by a 30\% price correction, and another even steeper price correction in the last weeks. After the correction, NVT has returned to the Green zone. This is another empiric evidence in support of 90 MA NVT.

There is, however, another more fundamental weakness of NVT. It only takes into account total value of on-chain transactions, but it does not factor in the number of transactions or the number of addresses (wallets) participating in these transactions. For internet companies, especially marketplaces, social networks, and other businesses with strong network effects, the analogous Daily Active Users (DAU) indicator is one of the most important performance and valuation metrics. This and other metrics that now make up the language of valuing internet companies did not exist in the 1990s. It has been developed by technology investors over the last 20+ years. Similar valuation framework for cryptoassets is yet to be developed and is only starting to form.

%----------------------------------------------------------------------------------------

\section{Solution and Benefits}
NVT Network is a prediction market and listing protocol that helps networks and creators better align with the quality of service they provide to users. This is accomplished by:

\begin{itemize}
\item A fair and efficient promotional system for creators and service providers.
\item User rewards for predicting quality products, services or networks. 
\item Simplifying the process of high quality discoveries.
\item Allowing all users to participate in content moderation and community decisions.
\end{itemize}

Users and communities can enjoy some benefits with NVT Network such as:

\begin{itemize}
\item Rewards for their effort and their work.
\item Rewards for discovering quality products and services.
\item Buying products or services in exchange for NVT tokens.
\item Explore high quality products more efficiently. 
\item Be part of content, data and governance moderation. 
\end{itemize}

From a high-level perspective, NVT Network has three key protocols that power these benefits:

\begin{itemize}
\item \textbf{Token Rewards Protocol.}

Solution to the unequal value distribution problem. A decentralized protocol for managing token rewards for quality work, to creators, users, and moderators while allowing and giving the incentive to users to spend tokens on the network.

\item \textbf{Product Pricing Protocol.}

Solution to the ineffective services and products discovery problem. A prediction and discovery market protocol for valuing, reviewing and rating services, products and networks.

\item \textbf{Community Protocol.}

Solution to the centralized moderation problem. The community protocol enables users to participate in content moderation and community decisions. Additionally, NVT Network is creating its own listing and sharing platform, NVT Community, to demonstrate and allow everyone to enjoy the benefits of NVT Network.

\end{itemize}

%----------------------------------------------------------------------------------------

\section{Information Discovery}

Based on the definition of ”Information”, the purpose of ”Information” is to eliminate uncertainty. Meanwhile the cost of getting information varies for different individuals. Facing complicated blockchain technology and booming blockchain currencies, regular investors are ”blinded” by the explosion of information. It is hard for one to get reliable information about a project with time-efficiency.

NVT Community solves the product-rating problem with the decentralized prediction market. Under this information filtering mechanism, quality products will stand out. NVT Community is powered by the blockchain, inheriting all the advantages of blockchain technology.

The developed payment system paved the ground for features like ”Paid Q\&A”, ”Paid Subscription”, ”Pay to message”, ”Pay to survey”, and ”Sending gift reward”. All those features encourage the distribution of valuable information, and paid information exchange.